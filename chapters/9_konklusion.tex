\section{Konklusion}
\subsection{Produkt}
Ved brug af den udleverede GUI har vi har fået udviklet et fungerende Monopoly-junior spil. Dette inkluderer alle de væsentlige ting fra det fysiske brætspil som spillernes figur, visuel terning, spillertransaktioner, fængsel og chancekort.

\subsection{Proces}
Inden opstart af projektet, fik vi i gruppen udarbejdet en gruppekontrakt. Denne kan findes i bilag \#2.
\\Til udviklingen brugte vi en udleveret GUI, som indeholdte meget af hvad, vi skulle bruge til spillet. Det er første gang, det har været fremgangsmåden, og har sådan set virket fint. Naturligvis krævede det, at vi lige skulle sætte os ind i dokumentationen for GUI'en. Det voldte dog heller ikke de store problemer.
\\
Dernæst har vi til rapportens udarbejdelse benyttet os af LaTeX i form af overleaf. Første indtryk var, at der ville være lidt af en indlæringskurve, hvilket også var tilfældet. Vi har styr på det mest grundliggende, men ved også, at der meget, som vi endnu mangler at få styr på. 
trel
\subsection{Perspektivering}
Selvom spillet kører som det skal med dets mest essentielle egenskaber, er der lige et par ting, som vi gerne ville have haft implementeret. Det er eksempelvis:
\begin{itemize}
\item Spiller vælger selv en unik brik-og farvekombination.
\item Få egne billeder og figurer visuelt vist på brættet.
\item Farven på huset på ejerens grunde matcher grundens ejer.
\item De resterende chancekort.
\item Dobbelt husleje hvis en pågældende spiller ejer begge felter af samme farve.
\item Landes der på et felt der allerede er ejet, og ejeren også ejer den anden ejendom i samme farve, fordobles beløbet der skal betales.
\item Hente alt tekst fra en txt. fil, så man nemt kan ændre eller oversætte spillet til et andet sprog.
\end{itemize}
