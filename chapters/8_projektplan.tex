\section{Projektplanlægning}
\subsection{Uge 45}
I uge 45 begynder vi projektet, ved at gøre os nogle tanker om hvordan det hele skal bygges op. Det er her vi planlægger hvornår hvad skal laves, så vi alle er klar over hvor der skal startes.\\
Vi begynder allerede her at kigge på de første UML modeller, samt specificerer vores krav.

\subsection{Uge 46}
Uge 46 vil vi udarbejde diverse modeller til at beskrive programmet. Vi vil blandt andet udarbejde en domæne model, diverse sekvensdiagrammer og system sekvens diagrammer. \\
Her vil vi også udarbejde nødvendige use cases, der skal til for at leve op vores krav.\\
Vi begynder på første dele af vores program i denne uge. Vi vil begynde at udarbejde de forskellige klasser og teste dem ved brug af JUnit tests.

\subsection{Uge 47}
I uge 47 vil vi færdiggøre vores kode og udføre de sidste test. 

\subsection{Uge 48}
Sidste uge vil vi samle projektet og færdiggøre rapporten. Her vil vi merche alt koden sammen. 


Vores spørgsmål vi ønsker somewhat besvaret er:
1. Ved at kigge UML diagrammerne igennem, kan I så forstå kodens opbygning? Kan de evt. forbedres, så det fremmer en bedre forståelse?
2. Giver koden overhovedet mening ift. en hvis mængde logik blandet sammet med GUIen i samme java class?
3. Er vores krav dækkende og tildels opfyldt ift. det leverede produkt?