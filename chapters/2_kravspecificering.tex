\subsection{Funktionelle krav}
\subsubsection{Opsætning}
\begin{enumerate}
  \item Spillerne skal kunne lande på et felt og så fortsætte derfra på næste slag. 
  \item Felterne er opbygget så spillerne går i ring på brættet.
  \item Spillet skal kunne spilles af 2 til 4 personer.
  \item Alle spillere skal starte med et antal penge
  ({\rotatebox[origin=c]{180}{\textwon}}).
  
  \begin{enumerate}
      \item Ved 2 spillere skal hver spiller have 20 {\rotatebox[origin=c]{180}{\textwon}}.
      \item Ved 3 spillere skal hver spiller have 18 {\rotatebox[origin=c]{180}{\textwon}}.
      \item Ved 4 spillere skal hver spiller have 16 {\rotatebox[origin=c]{180}{\textwon}}.
  \end{enumerate}
  
  \item Spillet skal tage brug af en GUI
  \begin{enumerate}
      \item Spillet skal i starten have et input felt, hvor brugeren skal angive antal personer.
      \item Derefter skal spillet skal have 1 knap, som bruges til kast af terningen.
      \item Spilleren skal kunne tage forskellige valg undervejs, når der trækkes et chancekort.
      \item GUI'en skal vise hver spillers pengebeholdning.
      \item GUI'en skal vise spillepladen med de forskellige felter og priser.
      \item GUI'en skal vise, hvor de enkelte spillere befinder sig på spillepladen.
      \item GUI'en skal vise, hvilke felter de enkelte spillere ejer.
  \end{enumerate}
  \item Alle felter der kan købes, skal have en pris fra 1 til 4 {\rotatebox[origin=c]{180}{\textwon}}.
\end{enumerate}



\smallskip
\subsubsection{Spilregler}
\begin{enumerate}
    \item Hver spiller kaster én terning, og rykker med uret rundt det antal felter på pladen, som øjnene viser.
    \item Hver gang der landes/passere start, modtager den enkelte spiller 2 {\rotatebox[origin=c]{180}{\textwon}}.
    \item Landes der på et ikke-ejet felt, skal spilleren købe det.
    \item Landes der på et felt der allerede er ejet, skal spilleren betale feltets værdi. Ejes feltet af samme spiller, gøres intet.
    \item Hvis en spiller lander på "gå i fængsel" feltet, bliver spilleren flyttet til fængselsfeltet uden at passere start. Derudover mister spilleren sin næste tur, og bliver derfor sprunget over en hel omgang.
    \item Hvis man lander på et "chance" felt, bliver et chancekort trukket.
    
    \item Spillet slutter når en modstander er gået fallit. 
    \item Når spillet slutter findes vinderen, som er den spiller med flest {\rotatebox[origin=c]{180}{\textwon}}.
\end{enumerate}



\bigskip



\subsection{Ikke funktionelle krav}
\begin{enumerate}
\subsubsection{Visuelt}
     \item GUI'en skal have en høj grad af visualitet og brugervenlighed
    \begin{enumerate}
        \item Det skal være nemt for alle spillere, at finde ud af, hvordan de interagerer med spillet.
        \item Det skal være nemt for brugerne, at trykke på knappen.
        \item Brugerne skal tydeligt kunne se, hvor mange {\rotatebox[origin=c]{180}{\textwon}} de har.
        \item Det skal være tydeligt for alle spillere at se, hvor samtlige spillere befinder sig på pladen.
        \item Det skal være tydeligt for brugeren at se, hvor mange {\rotatebox[origin=c]{180}{\textwon}} hver grund koster.
        \item Det skal være tydeligt for brugeren at se, hvilke grunde er til salg.
    \end{enumerate}
    \item 
    
\end{enumerate}
