\section*{Timeregnskab}

Alt timeregnskab er angivet i minutter.\\
Aflevering af opgave er sket Fredag uge 48.
\begin{table}[H]
\centering
\begin{tabular}{@{}lccccccc@{}} %c = center
\toprule
Uge 45   & Mandag & Tirsdag & Onsdag & Torsdag & Fredag & Lørdag & Søndag \\ \midrule
Frederik &    &    &    &  80  &    &    &    \\
Jean     &    &    &    &  80  &    &    &    \\
Marcus   &    &    &    &  80  &  30  &  30  &    \\
Rasmus   &    &    &    &  80  &    &    &    \\
Victor   &    &    &    &  80  &    &    &    \\
Villads  &    &    &    &      &    &    &    \\ 
\end{tabular}

\begin{tabular}{@{}lccccccc@{}}
\toprule
Uge 46   & Mandag & Tirsdag & Onsdag & Torsdag & Fredag & Lørdag & Søndag \\ \midrule
Frederik &  140  &  120  &    &    &    &    &    \\
Jean     &  140  &  120  &    &    &    &    &    \\
Marcus   &  140  &  100  &    &    &    &    &    \\
Rasmus   &  140  &  120  &    &    &    &    &    \\
Victor   &  140  &  120  &  60  &    &    &    &    \\
Villads  &    &   120 &    &    &    &    &    \\ 
\end{tabular}

\begin{tabular}{@{}lccccccc@{}}
\toprule
Uge 47   & Mandag & Tirsdag & Onsdag & Torsdag & Fredag & Lørdag & Søndag \\ \midrule
Frederik &    &    &    &    &    &    &    \\
Jean     &  120  &    &    &    &    &  60  &    \\
Marcus   &  180  &  60  &    &  120  &  300  &  70  &    \\
Rasmus   &    &    &    &    &  300  &    &    \\
Victor   &    &    &    &    &  300  &  70  &    \\
Villads  &    &    &    &    &    &    &    \\ 
\end{tabular}

\begin{tabular}{@{}lccccccc@{}}
\toprule
Uge 48   & Mandag & Tirsdag & Onsdag & Torsdag & Fredag & Lørdag & Søndag \\ \midrule
Frederik &  180  &    &    &120    &    &    &    \\
Jean     &  40  &    &    &   120 &    &    &    \\
Marcus   &  270 &    &    &  120  &  10  &    &    \\
Rasmus   &  270 &    30 &    &  120  &    &    &    \\
Victor   &  270 &    &  60  &  120  &    &    &    \\
Villads  &    &    &    &    &    &    &    \\ \bottomrule
\end{tabular}
\caption{Timeregnskab}
\end {table}

\clearpage


\section*{Resumé}

Formålet med CDIO 3 projektet er at udvikle det velkendte Monopoly Junior spil. Udviklerne har fået friheden til at selv bestemme hvilke regler der skal medtages, dog med brug af en udgivet GUI. 

I rapporten er der først og fremmest blevet identificeret krav og herefter er de enkelte krav opdelt efter funktionelle og ikke-funktionelle krav. Herefter er der blevet udarbejdet en analysdokumentation af use cases og flere forskellige typer af diagrammer. Spillets enkelte klasser er blevet uddybet og der er 
taget brug af den udleverede GUI. Slutteligt er der udført flere nødvendige tests af programmet og konfiguration - herunder importering fra Git. \\

\bigskip

Vores versionsstyringsrepository over rapporten kan findes ved følgende link:\\ https://github.com/marcusottosen/CDIO3overleaf\\

Vores repository til vores matadorspil kan findes ved følgende link:\\
https://github.com/marcusottosen/CDIO3




