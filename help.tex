% Dette dokument er kun til hjælp til overleaf!



%--- Oprettelse af nyt kapitel ---%
% Der oprettes en ny side pr kapitel man skriver. Kapitlet bliver først vist når der er blevet henvist til det i main.tex.
Start siden ud med \section{navn på chapter}


%--- Sektioner ---% (Den holder selv styr på numrene) Det der skrives i {} er titlen!
1	\section{section}               % 1
2	\subsection{subsection}         % 1.1
3	\subsubsection{subsubsection}   % 1.1.1
4	\paragraph{paragraph}
5	\subparagraph{subparagraph}


%Skråskift, Bold etc
\emph{din text} %italic

%-- line & page break --%
\\          % skifter linje (kan også bruge \linebreak
\smallskip  % laver mellemrum mellem linjer
\medskip
\bigskip
\linebreak[0-4] % Rykker linjen 0 til 4 linjer ned. Angiv kun 1 tal :)

\vfil       %rykker resterende tekst så langt ned på siden som muligt

\clearpage  % Rydder resten af siden og rykker hen på den næste
\newpage    % Brug denne i stedet for \clearpage


%-- Citer en kilde --%
Skriv kilden ind i document.bib 
Der findes @article, book, online, inbook og maange flere. Se link nedenfor.
f.eks. 
@online{knuthwebsite,
    author = "Donald Knuth",
    title = "Knuth: Computers and Typesetting",
    url  = "http://www-cs-faculty.stanford.edu/~uno/abcde.html",
    addendum = "(accessed: 01.09.2016)",
    keywords = "latex,knuth"
}
%https://www.overleaf.com/learn/latex/Bibliography_management_in_LaTeX#The_bibliography_file

Citer den i teksten ved:
\cite{china_too_many}
eller 
\shortcite{china_too_many}


//Ved oprettelse af dokument, sæt "\addbibresource{document.bib}" ind i toppen.
//\printbibliography printer bibliografien


%-- Figur/billede --%
% Et billede lægges ind i "images" mappen. Figurer lægges ind i figures.
\begin{figure}[H]
\centering
\includegraphics[width=4cm]{navn på fil fra figures mappen}
\caption{figurtekst her}
\end{figure}


%Man kan også tilføje kilde på figuren ved f.eks.:
\caption{meget flor billede. \cite{kildehenvisning }}


%-- Table/diagram  --% Eksempel:
https://www.tablesgenerator.com/

\begin{table}[H]
    \centering
    \caption{That is a table.}
    \begin{tabular}{|c|c|c|}
    \centering
    \textbf{Lorem} & \textbf{Ipsum} & \textbf{Lorem [\$]} \\ \hline \hline
1   & lipsum    & lorem     \\ \hline   
2   & lipsum    & lorem   \\ \hline
3   & lipsum    & lorem  
    \end{tabular}
    \label{tab:table1}
\end{table}


